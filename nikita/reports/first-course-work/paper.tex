\documentclass[aps,%
12pt,%
final,%
oneside,
onecolumn,%
musixtex, %
superscriptaddress,%
centertags]{article} %%
\topmargin=-40pt
\textheight=650pt
\usepackage[english, russian]{babel}
\usepackage[utf8]{inputenc}
%всякие настройки по желанию%
\usepackage[colorlinks=true,linkcolor=blue,unicode=true]{hyperref}
\usepackage{euscript}
\usepackage{supertabular}
\usepackage[pdftex]{graphicx}
\usepackage{amsthm, amssymb, amsmath}
\usepackage{textcomp}
\usepackage[noend]{algorithmic}
\usepackage[ruled]{algorithm}
\usepackage{pifont}
\selectlanguage{russian}

\begin{document}

\begin{titlepage}
\begin{center}
% Upper part of the page
\textbf{\Large САНКТ-ПЕТЕРБУРГСКИЙ \\ ГОСУДАРСТВЕННЫЙ УНИВЕРСИТЕТ} \\[1.0cm]
\textbf{\large Математико-Механический факультет} \\[0.2cm]
\textbf{\large Кафедра информационно-аналитических систем}\\[3.5cm]

% Title
\textbf{\LARGE Применение методов машинного обучения для автоматической разметки результатов геофизического исследования скважин}\\[1.0cm]
\textbf{\Large Курсовая работа студента 546 группы} \\[0.2cm]
\textbf{\Large Чурикова Никиты Сергеевича} \\[3.5cm]

%supervisor
\begin{flushright} \large
\emph{Научный руководитель:} \\
Доцент \textsc{Графеева Н. Г.}
\end{flushright}
\begin{flushright} \large
\emph{Заведующая кафедрой:} \\
Доцент \textsc{Михайлова Е. Г.}
\end{flushright}
\vfill

% Bottom of the page
{\large {Санкт-Петербург}} \par
{\large {2017 г.}}
\end{center}
\end{titlepage}

% Table of contents
\tableofcontents

\section{Введение}
Машинное обучение проникает во многие сферы нашей жизни~\cite{overview-of-ml} автоматизируя различные рутинные процессы, вроде поездок на машине и обработки рутинных документов. Поэтому у профессионалов из различных областей естественно возникает желание сократить время работы на не столь увлекательных задачах.

В данном тексте пойдет речь о применении машинного обучения в области геофизики. У специалистов в этой области есть очень трудоемкая задача по выделению на различной глубине в почве пород, основываяюсь на так называемых методах каротжа. Будет показано, что представляют из себя данные скважин, которые геофизики анализируют, какие наработки, продукты и технологии в данной области уже есть, а также будут приведены наработки и идеи автора по данной задаче.

\section{Обзор литературы}
Идея применения методов машинного обучения к задаче выделения пород в скважине не новая. Существует достаточно много литературы и статей на эту тему.

Начать разбираться в области применения машинного обучения к классификации литологии стоит с соревнования по данному вопросу~\cite{SEG-contest}, которое проводилось сообществом SEG~\cite{SEG}. В этом контесте приводят отличный пример того, как начинать с работать данными по скважинам. Также благодаря этому конкурсу, существует открытый датасет с разметкой пород. Они также объясняют и показывают, что породы бывают трудноразличимыми и потому ошибка в одну похожую породу допустима.

Также по результатам этого соревнования были написаны интересные статьи, которые кратко описывают научные результаты контеста. Статья~\cite{Bestagini2017a} подводит итоги и рассказывает о том, как генерировать новые атрибуты. Помимо стандартных подходов, вроде попарных перемножений фичей, они также предлагают считать градиент от атрибутов, воспринимая фичу, как функцию от глубины. Данный подход оказался достаточно удачным и был использован во всех лучших решениях. Помимо этого, работа показывает, что, что лучшим алгоритмом соревнования были деревья основанные на градиентном бустинге~\cite{xgboost}.

В статье~\cite{Tschannen2017} приведена попытка применить популярный алгоритм \\ convolutional neural network (CNN)~\cite{NIPS2012_4824} к данным соревнивания. Но, \\ несмотря на то, что они популярны, и то что атрибуты являются вещественными значениями, на этих данных алгоритм не попал даже в десятку лучших решений. Авторы статьи утверждают, что проблема заключается в недостаточном количестве данных.

Неплохой литературой для начала погружения в геофизику и машинное обучение является книга Мухамедиева~Р.И.~\cite{GeophysicsMLBook}. В этой работе приведено хороше описание методов каротажа, базовых алгоритмов машинного обучения, а также приводятся рекомендации по подготовке таких специфичных данных. В частности, они не рекомендуют использовать вейвлет преобразования~\cite{Mallat:2008:WTS:1525499}, а советуют обратить внимание на следующие этапы предобработки данных:
\begin{enumerate}
  \item Удаление аномальных значений;
  \item Линейная нормировка;
  \item Очистка данных по методу «ближайших соседей»;
  \item Формирование плавающего окна данных.
\end{enumerate}

\section{Постановка задачи}
Дана информация об обработке одной или нескольких скважин в некотором месторождении и известно, что на глубине $d^j_i$ встретилась порода $y^j_i$, где $j$ -- номер скважины. Также для каждой скважины $j$ и для каждой глубины $i$ известны значения применявшихся методов исследования скважин $x^j_i$ -- \textbf{методов каротажа}.

По данным $X$ необходимо сделать прогноз в новой скважине о том, какие породы в ней встретились для каждой глубины $i$.

Получается, что данную задачу можно интерпретировать, как задачу \textit{классификации}, где $X$ -- наблюдаемые значения, а $y$ -- целевая переменная.

\section{Решение}
В данной статье были использованы размеченные данные, полученные во время геофизических работ по исследованию угольных месторождений в республике Коми. Всего в датасете 216 скважин из трех месторождений. Для исследования, были использованы только первое месторождение, в котором 24 скважины.

\subsection{Разбиение на обучающую, валидационную и тестовую выборки}
Поскольку специфика применения исследования в том, чтобы применить исторические данные к новым, то было принято решение разбить данные из следующим образом:
\begin{description}
  \item [(train set)] В тренировочной выборке, все скважины из первого месторождения исключая одну
  \item [(dev set)] В валидационной выборке, одна случайная скважина из первого месторождения
  % \item [(test)] В тестовой выборке, второе месторождение
\end{description}

\subsection{Анализ целевой переменной}

% TODO: Возможно, стоит добавить, что за методы используются для балансировки классов
На Рис.~\ref{lithology-distribution-first} представлена целевая переменная в первом месторождении. Как видно на изображении, классы несбалансированы, что вносит определенные трудности в работу с ними. Для решения этой проблемы существует различное множество методов, однако мы ограничимся тем, что внутри классификатора будем понижать важность тех классов, которые представлены в большом количестве.

% TODO: Стоит повысить размер текста на изображениях
\begin{figure}[h!]
\center
  \scalebox{0.3}{
    \includegraphics{../../log-images/first-deposit-target-values.png}
  }
  \caption{Породы в первом месторождении}
  \label{lithology-distribution-first}
\end{figure}

% TODO: Добавить распределение пород во втором месторождении
% \begin{figure}
%   \scalebox{0.3}{
%     \includegraphics{../../log-images/second-deposit-target-values.png}
%   }
%   \caption{Породы в первом месторождении}
%   \label{lithology-distribution-second}
% \end{figure}

Существует также еще одна потенциальная проблема с размеченными данными. Проблема состоит в том, что их размечает некоторый эксперт. Человек. Т.е. он вносит определенный шум, который определяется его усталостью, неопытностью, ленью - то что называется \textit{Человеческий фактор}. К сожалению, мы пока никак не можем повлиять на это, поскольку в нашем распоряжении имеется лишь эта информация.

Хотя в будущем, при наличии денег и ресурсов, было бы великолепно попросить $n$ геофизиков разметить эти данные и оценить разброс их ответов.

\subsection{Анализ наблюдаемых значений}
Наблюдаемыми значениями в данной задаче являются значения методов каротажа, которые использовались в скважине. Их существует огромное количество~\cite{carrot-methods-wiki}, и при различных ситуациях могут использоваться различные их комбинации. В нашем датасете используется от трех до пяти методов. В приложении 1 приведено описание этих методов.

% TODO: Нужно доказательство
Логично предположить, что чем больше мы проведем геофизических методов исследований, тем более точные данные получатся. Именно поэтому и используют сейчас больше одного метода. Но подобная практика создает сложности для обучения алгоритмов машинного обучения, поскольку текущие алгоритмы не способны работать с пропущенными атрибутами. Эта проблема является отдельной темой исследования. На данном этапе мы ограничимся тем, что будем использовать только те методы, которые есть в обоих месторождениях.

Еще одна проблема с данными, это то что они поступают с датчика, и потому являются шумными данными. В данной работе с этим не было никаких попыток борьбы.

\subsection{Классификация}
Для классификации было принято решение использовать популярный алгоритм Random Forest~\cite{RandomForest} и Логистическую регрессию~\cite{LogisticRegression}. По результатам классификации получилась следующая матрица неточности, на \textit{dev set}.

\begin{figure}[h!]
\center
  \scalebox{0.4}{
    \includegraphics{../../log-images/predictions/RandomForest_n_estimators-10_max_depth-8_class_weight-balanced_random_state-30-2017-11-5-16-52-18.png}
  }
  \caption{Random Forest с n\_estimators=10, max\_depth=8, class\_weight='balanced'}
  \label{Random Forest clf}
\end{figure}

\begin{figure}[h!]
\center
  \scalebox{0.4}{
    \includegraphics{../../log-images/predictions/LogisticRegression_class_weight-balanced_random_state-30-2017-11-5-19-23-13.png}
  }
  \caption{Logistic Regression с class\_weight='balanced'}
  \label{Logistic Regression clf}
\end{figure}

По результатам классификации, которые можно увидеть на Рис.~\ref{Random Forest clf},~\ref{Logistic Regression clf}, получается, что довольно неплохо выделяются уголь, алевролит и песчаник.

С алевролитом и песчаником довольно понятная ситуация. Они представлены в датасете в большом количестве и потому их важность завышается.

С углем можно предположить следующие причины того хорошего предсказания:
\begin{itemize}
  \item Он является самой интересной породой для исследователей и потому ему уделяется особое внимание;
  \item Также под него могут подбирать методы каротажа, которые его хорошо выделяют.
\end{itemize}

\section{Выводы}
Дипломная работа должна будет содержать следующее:
\begin{description}
  % \item [\ding{51}] Найти данные угольных скважин
  \item [±] Исследование данных угольных скважин
  \begin{itemize}
    \item Исследована на данный момент только одна скважина
  \end{itemize}
  \item [+] Проверка концепции, что можно прогнозировать породу, используя исторические данные о месторождении.
  \item [±] Применение различные методы нормализации данных;
  \begin{itemize}
    \item При обучении Logistic Regression было использовано вычитание среднего и деление на дисперсию
  \end{itemize}
  \item [-] Применение других методов классификации
  \item [-] Рассмотрена идея того, что методы каротажа могу быть нелинейно зависимы друг от друга. Возможно, эти зависимости можно будет выделить с помощью нелинейных алгоритмов регрессии или нелинейных функций преобразований;
  \item [-] Генерация новых признаков посредством их различного комбинирования (перемножение, подсчет градиента, и т.п.)
  \item [-] Использована и рассмотрена подробно остальная часть датасета;
\end{description}

\section{Заключение}
Данная работа является очень актуальной, поскольку данные методы могут помочь геофизикам точнее размечать породы в земле. Были представлены базовые идеи, от которых можно отталкиваться. Несмотря на не очень впечатляющие результаты классификации, алгоритмы машинного обучения способы выделять породы без априорного знания о данных, если есть замеченные данные.


\bibliographystyle{gost780s}
\bibliography{test}

\section{Приложение 1}
\begin{itemize}
  \item \textbf{КС} — кажущееся сопротивление с нефокусированными зондами. Самый распространённый метод данной группы, являющийся скважинным аналогом метода электрического профилирования в электроразведке
  \item \textbf{резистивиметрия (REZ)} С помощью этого метода измеряют удельное электрическое сопротивление жидкости, заполняющей в данный момент скважину. Жидкость может быть представлена как буровым раствором (его сопротивление заранее известно), так и пластовыми флюидами (нефть, пресная или минерализованная вода), а также их смесью
  \item \textbf{БК} — боковой каротаж. Отличие от классического КС заключается в фокусировке тока зондом
  \item \textbf{ГК} — гамма-каротаж. Очень простой и распространённый метод, измеряющий только естественное гамма-излучение от пород, окружающих скважину. Существует его чуть более усложнённый вариант — спектрометрический гамма-каротаж (СГК или ГК-С), который позволяет различить попавшие в детектор геофизического зонда гамма-кванты по их энергии. По этому параметру можно точнее судить о характере слагающих толщу пород.
  \begin{itemize}
    \item Основная расчетная величина – мощность экспозиционной дозы в микрорентгенах в час (МЭД, мкР/ч). Измеряемая величина определяется концентрацией, составом и пространственным распределением ЕРЭ, плотностью $\rho$ и эффективным атомным номером Zэфф пород.
    \item Входит в число обязательных методов
  \end{itemize}
  \item \textbf{ГГК} — гамма-гамма каротаж. Геофизический зонд облучает породу гамма-излучением, в результате которого порода становится радиоактивной и в ответ тоже излучает гамма-кванты. Именно эти кванты и регистрируются зондом.
\end{itemize}

\end{document}
