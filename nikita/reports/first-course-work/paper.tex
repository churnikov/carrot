\documentclass[aps,%
12pt,%
final,%
oneside,
onecolumn,%
musixtex, %
superscriptaddress,%
centertags]{article} %%
\topmargin=-40pt
\textheight=650pt
\usepackage[english, russian]{babel}
\usepackage[utf8]{inputenc}
%всякие настройки по желанию%
\usepackage[colorlinks=true,linkcolor=blue,unicode=true]{hyperref}
\usepackage{euscript}
\usepackage{supertabular}
\usepackage[pdftex]{graphicx}
\usepackage{amsthm, amssymb, amsmath}
\usepackage{textcomp}
\usepackage[noend]{algorithmic}
\usepackage[ruled]{algorithm}
\selectlanguage{russian}

\begin{document}

\begin{titlepage}
\begin{center}
% Upper part of the page
\textbf{\Large САНКТ-ПЕТЕРБУРГСКИЙ \\ ГОСУДАРСТВЕННЫЙ УНИВЕРСИТЕТ} \\[1.0cm]
\textbf{\large Математико-Механический факультет} \\[0.2cm]
\textbf{\large Кафедра информационно-аналитических систем}\\[3.5cm]

% Title
\textbf{\LARGE Применение методов машинного обучения для автоматической разметки результатов геофизического исследования скважин}\\[1.0cm]
\textbf{\Large Курсовая работа студента 546 группы} \\[0.2cm]
\textbf{\Large Чурикова Никиты Сергеевича} \\[3.5cm]

%supervisor
\begin{flushright} \large
\emph{Научный руководитель:} \\
Доцент \textsc{Графеева Н. Г.}
\end{flushright}
\begin{flushright} \large
\emph{Заведующая кафедрой:} \\
Доцент \textsc{Михайлова Е. Г.}
\end{flushright}
\vfill

% Bottom of the page
{\large {Санкт-Петербург}} \par
{\large {2017 г.}}
\end{center}
\end{titlepage}

% Table of contents
\tableofcontents

\section{Введение}
Машинное обучение проникает во многие сферы нашей жизни~\cite{overview-of-ml} автоматизируя различные рутинные процессы, вроде поездок на машине~\cite{ai-cars} и обработки рутинных документов. Поэтому у профессионалов из различных областей естественно возникает желание сократить время работы на не столь увлекательных задачах.

В данном тексте пойдет речь о применении машинного обучения в области геофизики. У специалистов в этой области есть очень трудоемкая задача по выделению на различной глубине в почве пород, основываяюсь на так называемых методах каротжа. Будет показано, что представляют из себя данные скважин, которые геофизики анализируют, какие наработки, продукты и технологии в данной области уже есть, а также будут приведены наработки и идеи автора по данной задаче.

\section{Обзор литературы}
Идея применения методов машинного обучения к задаче выделения пород в скважине не новая. Существует достаточно много литературы и статей на эту тему.

Начать разбираться в области применения машинного обучения к классификации литологии стоит с соревнования по данному вопросу~\cite{SEG-contest}, которое проводилось сообществом SEG~\cite{SEG}. В этом контесте приводят отличный пример того, как начинать с работать данными по скважинам. Также благодаря этому конкурсу, существует открытый датасет с разметкой пород. Они также объясняют и показывают, что породы бывают трудноразличимыми и потому ошибка в одну похожую породу допустима.

Также по результатам этого соревнования были написаны интересные статьи, которые кратко описывают научные результаты контеста. Статья~\cite{Bestagini2017a} подводит итоги и рассказывает о том, как генерировать новые атрибуты. Помимо стандартных подходов, вроде попарных перемножений фичей, они также предлагают считать градиент от атрибутов, воспринимая фичу, как функцию от глубины. Данный подход оказался достаточно удачным и был использован во всех лучших решениях. Помимо этого, работа показывает, что, что лучшим алгоритмом соревнования были деревья основанные на градиентном бустинге~\cite{xgboost}.

В статье~\cite{Tschannen2017} приведена попытка применить популярный алгоритм convolutional neural network (CNN)~\cite{cnn} к данным соревнивания. Но, несмотря на то, что они популярны, и то что атрибуты являются вещественными значениями, на этих данных алгоритм не попал даже в десятку лучших решений. Авторы статьи утверждают, что проблема заключается в недостаточном количестве данных.

Неплохой литературой для начала погружения в геофизику и машинное обучение является книга Мухамедиева~Р.И.~\cite{GeophysicsMLBook}. В этой работе приведено хороше описание методов каротажа, базовых алгоритмов машинного обучения, а также приводятся рекомендации по подготовке таких специфичных данных. В частности, они не рекомендуют использовать вейвлет преобразования~\cite{wavelet}, а советуют обратить внимание на следующие этапы предобработки данных:
\begin{enumerate}
  \item Удаление аномальных значений;
  \item Линейная нормировка;
  \item Очистка данных по методу «ближайших соседей»;
  \item Формирование плавающего окна данных.
\end{enumerate}

\section{Постановка задачи}
Дана информация об обработке одной или нескольких скважин в некотором месторождении и известно, что на глубине $d^j_i$ встретилась порода $y^j_i$, где $j$ -- номер скважины. Также для каждой скважины $j$ и для каждой глубины $i$ известны значения применявшихся методов исследования скважин $x^j_i$ -- \textbf{методов каротажа}.

По данным $X$ необходимо сделать прогноз в новой скважине о том, какие породы в ней встретились для каждой глубины $i$.

Получается, что данную задачу можно интерпретировать, как задачу \textit{классификации}, где $X$ -- наблюдаемые значения, а $y$ -- целевая переменная.

\section{Решение}

В данной статье были использованы размеченные данные, полу

\subsection{Анализ данных угольных шахт}

В рамках проверки концепции, что можно делать прогноз о типе литологии по данным каротажа, были использованы данные, полученные в результате исследования наличия угля в месторождении. Нам были предоставлены данные по двум месторождениям: 24 скважины из первого месторождения и 7 скважин из второго. На обоих месторождениях использовались, в большинстве своем, одинаковые методы каротажа, однако есть различия два метода.
% TODO: Если успею, то прикрутить аппроксимацию обоих методов, если нет, то просто сказать, что это классно было бы сделать потом

На Рис.~\ref{first-deposit-visualisation} можно увидеть визуализацию первого месторождения, используя метод понижения размерности TSNE~\cite{tsne-basic}. Видно, что некоторые литологии можно выделить разделяющей поверхностью, но также некоторые встречаются значительно чаще других, а некоторые литологии вовсе не видно. Таким образом, встает проблема \textit{несбалансированных классов}, что подтверждается также графиком распределния классов.
% TODO: Добавить график распределния классов.
\begin{figure}[ht]
\begin{center}

\scalebox{0.3}{
  \includegraphics{../../log-images/decomposition/decomposition-of-first-deposit-scaled-feats-bk-ggk-gr-ks.png}
}

\caption{
\label{first-deposit-visualisation}
     Визуализация данных каротажа в двухмерном пространстве с использованием метода TSNE.}
\end {center}
\end {figure}

\begin{figure}
  \scalebox{0.3}{
    \includegraphics{../../log-images/first-deposit-target-values.png}
  }
  \caption{Распределение литологий}
  \label{lithology-distribution}
\end{figure}



% \subsection{Мотивация}
% \subsection{Постановка задачи}
% \subsection{Доступные программные средства}
% \subsection{Полученные результаты}
%
% \section{Основная часть раз}
% Секций в основной части может быть сколько угодно.
%
% \section{Основная часть два: Теория}
%
% \section{Основная часть два: Детали реализации}
% \subsection{Расчётная часть}
%
% \section{Анализ экспериментов.}
% \begin{figure}[ht]
% \begin{center}
%
% \scalebox{0.4}{
%    \includegraphics{images/graph.jpg}
% }
%
% \caption{
% \label{graph-fig}
%      Линейные функции.}
% \end {center}
% \end {figure}
% Ссылаемся на график ~\ref{graph-fig}.
% Ссылка на статью: \cite{DBLP:conf/adbis/NovikovP03}
\section{Заключение}

%ручной ввод библиографии.
%для тестирования, убрать комментарий
%\begin{thebibliography}{}

%\bibitem{voc} Griffin D.W., Lim J.S. \flqq Multiband excitation vocoder\frqq. IEEE ASSP-36 (8), 1988, pp. 1223-1235.
%\end{thebibliography}

%автоматическая генерация библиографии из бибтеховского файла. Из файла подгрузятся только те статьи, на которые есть ссылки в тексте!
\bibliographystyle{gost780s}
\bibliography{test}

\end{document}
